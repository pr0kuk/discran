\documentclass{article}
\usepackage[utf8]{inputenc}
\usepackage{amsmath}
\usepackage[T2A]{fontenc}
\usepackage[russian]{babel}
\usepackage{mathtext}
\title{Task №1}
\author{Щербаков Алексей Б01-908}
\date{16 September 2019}
\begin{document}
\maketitle
\section{}
x,y,z $-$ целые числа, для которых истинно высказывание \\
\begin{math}
\neg(x = y)\wedge((y<x)\rightarrow(2z>x))\wedge((x<y)\rightarrow(x>2z))
\end{math}
\\Чему равно $x$, если $z=7, y=16$?
\\Решение
\\Если высказывание истинно, то:
\begin{equation*}
 \begin{cases}
   x\neq y\\
   ((y<x)\rightarrow(2z>x))\\
   ((x<y)\rightarrow(x>2z))\\
 \end{cases}
\end{equation*}
Если $x>16$, то $y<x$ и $2z<x$, следовательно, второе выражение ложно.\\
Если $x<14<y$, то $2z>x$, следовательно, третье выражение ложно.\\
Если $x=15$, то $x<y$ и $x>2z$, следовательно все выражения истинны.
\\Ответ: 15



\section{}
Постройте таблицу истнности для $\neg((x\wedge \neg y) \wedge z)\\$
\begin{center}
\begin{tabular}{ |c|c|c|c| } 
 \hline
 x & y & z & res \\
 0 & 0 & 0 & 1 \\ 
 0 & 0 & 1 & 1 \\ 
 0 & 1 & 0 & 1 \\ 
 0 & 1 & 1 & 1 \\
 1 & 0 & 0 & 1 \\ 
 1 & 0 & 1 & 0 \\
 1 & 1 & 0 & 1 \\ 
 1 & 1 & 1 & 1 \\ 

 \hline
\end{tabular}
\end{center}

\section{}
Докажите, что\\
\begin{math}
1 \oplus x_1 \oplus x_2 = (x_1 \rightarrow x_2) \wedge (x_2 \rightarrow x_1)
\end{math}
\\Построим таблицы истинности
\begin{center}
\begin{tabular}{ |c|c|c| } 
 \hline
 x_1 & x_2 & res \\
 0 & 0 & 1 \\ 
 0 & 1 & 0 \\ 
 1 & 0 & 0 \\ 
 1 & 1 & 1 \\ 
 \hline
\end{tabular}
\begin{tabular}{ |c|c|c| } 
 \hline
 x_1 & x_2 & res \\
 0 & 0 & 1 \\
 0 & 1 & 0 \\ 
 1 & 0 & 0 \\ 
 1 & 1 & 1 \\ 
 \hline
\end{tabular}
\end{center}
Таблицы совпадают, следовательно равенство выполняется.


\section{}
Выполняется ли дистрибутивность для следующих операций:\\
а)
\begin{math}
x\wedge(y\rightarrow z) ? (x\wedge y) \rightarrow (x\wedge z)
\end{math}
\begin{center}
\begin{tabular}{ |c|c|c|c| } 
 \hline
 x & y & z & res \\
 0 & 0 & 0 & 0 \\ 
 1 & 0 & 0 & 1 \\ 
 1 & 1 & 0 & 0 \\ 
 1 & 1 & 1 & 1 \\ 
 0 & 1 & 0 & 0 \\ 
 0 & 0 & 1 & 0 \\ 
 0 & 1 & 1 & 0 \\
 1 & 0 & 1 & 0 \\
 \hline
\end{tabular}
\begin{tabular}{ |c|c|c|c| } 
 \hline
 x & y & z & res \\
 0 & 0 & 0 & 1 \\ 
 1 & 0 & 0 & 1 \\ 
 1 & 1 & 0 & 0 \\ 
 1 & 1 & 1 & 1 \\ 
 0 & 1 & 0 & 1 \\ 
 0 & 0 & 1 & 1 \\ 
 0 & 1 & 1 & 1 \\
 1 & 0 & 1 & 0 \\
 \hline
\end{tabular}
\end{center}
Таблицы не совпадают, следовательно дистрибутивноть не выполняется.

\\
б) $x\oplus (y \leftrightarrow z) ? (x \oplus y) \leftrightarrow (x \oplus z)$
\begin{center}
\begin{tabular}{ |c|c|c|c| } 
 \hline
 x & y & z & res \\
 0 & 0 & 0 & 1 \\ 
 1 & 0 & 0 & 0 \\ 
 1 & 1 & 0 & 1 \\ 
 1 & 1 & 1 & 0 \\ 
 0 & 1 & 0 & 0 \\ 
 0 & 0 & 1 & 0 \\ 
 0 & 1 & 1 & 1 \\
 1 & 0 & 1 & 1 \\
 \hline
\end{tabular}
\begin{tabular}{ |c|c|c|c| } 
 \hline
 x & y & z & res \\
 0 & 0 & 0 & 1 \\ 
 1 & 0 & 0 & 1 \\ 
 1 & 1 & 0 & 0 \\ 
 1 & 1 & 1 & 1 \\ 
 0 & 1 & 0 & 0 \\ 
 0 & 0 & 1 & 0 \\ 
 0 & 1 & 1 & 1 \\
 1 & 0 & 1 & 0 \\
 \hline
\end{tabular}
\end{center}
Таблицы не совпадают, следовательно дистрибутивноть не выполняется.\\
\section{}
а) Выполняется ли для импликации коммутативность?\\
Очевидно, что $x \rightarrow y$ не равно $y \rightarrow x$ (пример x = 1, y = 0)\\
б) Выполняется ли для импликации ассоциативность?\\
$(x\rightarrow y) \rightarrow z ? x \rightarrow (y \rightarrow z)$\\
Нет, пример $x=0, y=0, z=0$, левая часть 0, правая 1.\\
\section{}
Указать существенные и фиктивные переменные\\
а) $f(x1, x2, x3)=00111100$
\begin{center}
\begin{tabular}{ |c|c|c|c| } 
 \hline
 x_1 & x_2 & x_3 & res \\
 0 & 0 & 0 & 0 \\ 
 0 & 0 & 1 & 0 \\ 
 0 & 1 & 0 & 1 \\ 
 0 & 1 & 1 & 1 \\ 
 1 & 0 & 0 & 1 \\ 
 1 & 0 & 1 & 1 \\ 
 1 & 1 & 0 & 0 \\
 1 & 1 & 1 & 0 \\
 \hline
\end{tabular}
\end{center}
z - фиктивная, x и y - существенные

б) $g(x_1, x_2, x_3 = (x_1 \rightarrow (x_1 \vee x_2) \rightarrow x_3$
\begin{center}
\begin{tabular}{ |c|c|c|c| } 
 \hline
 x_1 & x_2 & x_3 & res \\
 0 & 0 & 0 & 0 \\ 
 0 & 0 & 1 & 1 \\ 
 0 & 1 & 0 & 1 \\ 
 0 & 1 & 1 & 1 \\ 
 1 & 0 & 0 & 0 \\ 
 1 & 0 & 1 & 1 \\ 
 1 & 1 & 0 & 0 \\
 1 & 1 & 1 & 1 \\
 \hline
\end{tabular}
\end{center}
x, y, z - существенные\\
\section{}
Докажите формулу разложения:\\
$f(x_1,...,x_n)=(x_1 \vee f(0, x_2, ..., x_n)) \wedge (\neg x_1 \vee f(1, x_2, ..., x_n))$
\\
Если $x_1=0$, то $(\neg x_1 \vee f(1, x_2, ..., x_n))=1, а (x_1 \vee f(0, x_2, ..., x_n))=f(x_1,...,x_n)$ чтд\\
Если $x_1=1$, то $(x_1 \vee f(0, x_2, ..., x_n))=1, а (\neg x_1 \vee f(1, x_2, ..., x_n))=f(x_1,...,x_n)$  чтд\\
\section{}
Если функция истинна, то каждая $x_n^\alpha=1$, следовательно при замене любого, хотя бы одного $\alpha$, данный x обращается в 0, и вся функция обращается в 0 чтд\\
\section{}
$(x_1 \vee x_2 \vee ... \vee x_n) \wedge (\bar x_1 \wedge \bar x_2 \wedge ... \wedge \bar x_n) = (x_1 \wedge \bar x_1) \vee (x_1 \wedge \bar x_2) ...\\ $Так как $(x_i \wedge \bar x_i) = 0$, то:\\
$(x_1 \vee x_2 \vee ... \vee x_n) \wedge (\bar x_1 \wedge \bar x_2 \wedge ... \wedge \bar x_n) = (x_1 \wedge \bar x_2) \vee (x_2 \wedge \bar x_1) \vee (x_2 \wedge \bar x_3) \vee (x_3 \wedge \bar x_2) \vee ... =\\= (x_1 \oplus x_2) \vee (x_2 \oplus x_3) \vee...)$ чтд.
\end{document}