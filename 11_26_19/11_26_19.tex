\documentclass{article}
\usepackage{amsfonts}
\usepackage{amssymb}
\usepackage[all]{xy}
\usepackage[utf8]{inputenc}
\usepackage{amsmath}
\usepackage[T2A]{fontenc}
\usepackage[russian]{babel}
\usepackage{mathtext}
\title{Task №11}
\author{Щербаков Алексей Б01-908}
\date{26 November 2019}
\begin{document}
\maketitle
\section{}
Известно, что в неориентированном графе существует маршрут, проходящий по каждому ребру ровно два раза. Верно ли, что в графе есть замкнутый эйлеров маршрут?\\\\
\xymatrix{
B \ar@{-}[r]\ar@{-}[rd]  & C\ar@{-}[d]\\
A \ar@{-}[r]\ar@{-}[u]  & D\\}\\\\
В этом графе существует маршрут удовлетворяющий условию: $ABDCBADCBDA$, но при этом замкнутного эйлерова маршрута не существует, так как существует вершина с нечётной степенью\\
Ответ: Неверно
\section{}
Выходная степень каждой вершины в ориентированном графе на $n$ вершинах равна $n-2$. Какое количество компонент сильной связности может быть в этом графе? Укажите все возможные значения.\\\\
Для 1 и 2 есть примеры:\\
\xymatrix{
B \ar@{-}[r] & C\ar@{-}[d] & & & B\ar@{-}[rd] & & D\ar@{->}[ld] \ar@{->}[ll]\\
A \ar@{-}[u]\ar@{-}[r]  & D & & A \ar@{-}[ru]\ar@{-}[rr] & & C\\}\\\\
Докажем, что больше двух быть не может:\\
Пусть есть больше одной компоненты сильной связности, тогда существует вершина $A$ такая, что не существует ребра из $A$ в $B$. Тогда вершина $A$ соединена со всеми остальными вершинами, которые в свою очередь соединены тоже со всеми вершинами кроме $B$. Таким образом, есть всего две компоненты связности. Вершина $B$ и все остальные вершины. Значит, если компонент сильной связности больше одной, то их две.
\\
Ответ: 1 и 2
\section{}
Пусть в ориентированном графе для любой пары вершин $u, v$ есть либо ребро $(u,v)$, либо ребро $(v,u)$ (ровно одно из двух). Докажите, что в таком графе есть (простой) путь, включающий в себя все вершины.\\\\
Докажем по индукции:\\
Для двух вершин верно\\
Пусть существует такой путь для $n$ вершин\\
\xymatrix{
A \ar@{->}[r] & B\ar@{->}[r] & C\ar@{->}[r] & ...\ar@{->}[r] & Z\\}\\\\
Докажем, что существует путь для $n+1$ вершин\\
Пусть $n+1$-я вершина - $X$.\\
Если существует ребро $(X,A)$ или $(Z,X)$, то очевидно верно\\
Тогда рассмотрим случай, когда существуют рёбра $(A,X)$ и $(X,Z)$\\
Для каждой из $n$ вершин ребро или входит из $X$ или уходит в $X$\\
Из вершины $A$ ребро уходит в $X$, а в $Z$ входит\\
Значит существует вершина $Q$ такая, что она и все предыдущие вершины имели рёбра вида $(P,X)$, а следующая имеет ребро вида $(X,P)$.\\
\xymatrix{
&&&&&X\ar@{->}[d]\ar@{->}[rrd]\\
A \ar@{->}[r]\ar@{->}[urrrrr] & B\ar@{->}[r]\ar@{->}[urrrr] & C\ar@{->}[r]\ar@{->}[urrr] & ... \ar@{->}[r] \ar@{->}[urr] & Q\ar@{->}[r]\ar@{->}[ur] & T\ar@{->}[r] & ...\ar@{->}[r]&Z\\}\\\\
Таким образом, существует путь \\
\xymatrix{
&&&&&X\ar@{->}[d]\\
A \ar@{->}[r] & B\ar@{->}[r] & C\ar@{->}[r] & ... \ar@{->}[r] & Q\ar@{->}[ur] & T\ar@{->}[r] & ...\ar@{->}[r]&Z\\}\\\\
чтд
\section{}
Профессор Рассеянный построил частичный порядок $<_P$ для утреннего одевания:\\
очки $<_P$ брюки $<_P$ ремень $<_P$ пиджак,\\
очки $<_P$ рубашка $<_P$ галстук $<_P$ пиджак,\\
брюки $<_P$ туфли,\\
очки $<_P$ носки $<_P$ туфли,\\
очки $<_P$ часы.\\
\\\\
а) Постройте линейный порядок на вещах, не нарушаемый исходный порядок\\
очки $<_P$ брюки $<_P$ носки $<_P$ туфли $<_P$ ремень $<_P$ рубашка $<_P$ галстук $<_P$ пиджак $<_P$ часы\\\\
б) Сколько всего существует таких линейных порядков?\\
Найдём количество способов разместить рубашку и галстук в порядок \\"очки $<_P$ брюки $<_P$ ремень $<_P$ пиджак"\\
Есть три места для первого элемента и четыре места для второго\\(итого $3\cdot4=12$), но так как рубашка $<_P$ галстук, то кол-во способов $\frac{12}{2} = 6$\\
Аналогично расставим носки и туфли (итого $\frac{6\cdot7}{2}=21)$\\
Теперь нужно учесть брюки $<_P$ туфли: вычтем из $21\cdot6=126$ количество способов где туфли стоят раньше брюк: $126-\frac{5\cdot6}{2}=111$.\\
И ещё 8 способок расставить часы, следовательно: $111\cdot8=888$\\\\
Ответ: 888

\section{}
Существует $n$ городов, каждые два соединены дорогой. Устанавливается одностороннее движение так, что если из города можно выехать, то в него нельзя вернуться. Докажите следующие утверждения.\\\\
1.а) так можно сделать\\
Пронумеруем города. Рассмотрим любые два гоорода и пустим одностороннее движение в сторону города с большим номером. В таком случае из города с большим номером невозможно вернуться в город с меньшим. чтд\\\\
1.б) найдётся город, из которого можно добраться до всех, и найдётся город, из которого нельзя уехать\\
Если из любого города можно было бы куда-нибудь уехать, можно было бы бесконечно передвигаться по этому графу, а так как количеству городов конечно, то города бы повторялись, противоречие с условием, значит есть город из которого нельзя уехать и он только один, так как в него входит $n-1$ ребро. Удалим этот город со всеми входящими в него рёбрами. После удаления опять должен остаться город из которого нельзя уехать. Будем так удалять города пока не останется один город без рёбер. Так как при процессе удаления из неудалённых городов удаляются только выходящие рёбра, то в этот город никогда не входили рёбра, значит существет город, в который нельзя въехать, причём только один.
\\\\
Аналогичным рассуждением можно получить, что ровно один город имеет $n-1$ выходящих рёбер, один город имеет $n-2$ выходящих рёбер и так вплоть до 0. Таким образом, оказывается, что существует принципиально только один способ установить односторонее движение, а различаются они только перестановкой городов.
\\\\
1.в) Так как существует только один принцип построения такого графа, то это построение по типу из пункта а). Тогда существует единственный маршрут, обходящий все города $1\rightarrow2\rightarrow...\rightarrow n$\\\\
2. Так как все способы отличаются только номерами городов, то $n!$\\\\
\section{}
Докажите, что либо турнир (антирефлексивное, антисимметричное, линейное бинарное отношение $P$) - строгий линейный порядок, либо существуют такие альтернативы $a,b,c$, что $aPb,bPc$ и $cPa$.
\\\\
Пусть не существуют такие альтернативы $a,b,c$, что $aPb,bPc$ и $cPa$, тогда отношение транзитивно. Так как турнир антирефлексивный, антисимметричный и транзитивный, то $P$ - строгое отношение частичного порядка. Так как $\forall x,y \in A: x\neq y \Rightarrow (xPy) \wedge (yPx)$, то $P$ - строгий линейный порядок чтд.
 такие альтернативы $a,b,c$, что $aPb,bPc$ и $cPa$, тогда отношение транзитивно. Так как турнир антирефлексивный, антисимметричный и транзитивный, то $P$ - строгое отношение частичного порядка. Так как $\forall x,y \in A: x\neq y \Rightarrow (xPy) \wedge (yPx)$, то $P$ - строгий линейный порядок чтд.
\section{}
Сколько есть порядков на $n$-элементном множестве, в которых ровно одна пара элементов несравнима?\\\\
Пусть $A_p$ и $A_q$ не сравнимы, тогда так как это единственная пара, то все элементы без любого из этих образуют линейный порядок. Пусть $A_p$ и $A_q$ расположены не между одними и теми же элементами, тогда элементы были бы сравнимы. (Например: Пусть $A_{p-1}<A_p<A_{p+1}, A_{p-3}< A_q < A_{p-1}$, тогда $A_q < A_{p-1}<A_p$ - противоречие). Значит $A_p$ и $A_q$ расположены между одними и теми же элементами в соответственных линейных порядках.\\
Выберем несравнимую пару: C$^2_n$\\
Зададим линейный порядок из оставшихся элементов: $(n-2)!$\\
Выберем место в линейном порядке куда вставить несравнимую пару: $n-1$\\
Итого: C$^2_n \cdpt (n-2)! \cdot n-1 = (n-1)!\text{C}^2_n = \frac{(n-1)n!}{2}$\\\\
Ответ: $\frac{(n-1)n!}{2}$
\section{}
Докажите, что любой частичный порядок $P$ на конечном множестве $A$ можно продолжить до линейного. То есть можно добавить в $P$ некоторые пары элементов из $A\times A$ так, что любые два элемента $a,b\in A$ окажутся сравнимы: будет выполнено либо $aPb$ либо $bPa$\\\\
Составим линейный порядок на множестве $A\times A$ следующим образом:\\
Возьмём два несравнимых элемента $p$ и $q$ из множества $A\times A$ и будем считать, что $p\leq q$, поместим в линейный порядок все элементы сравнимые с $p$ и $q$. Повторяем данную операцию до тех пор, пока не построим линейный порядок. Таким образом, мы построили какой-то линейный порядок на множестве $A\times A$, который содержит частичный порядок $P$, данный в условии. Значит, частичный порядок $P$ можно продолжить до линейного порядка, чтд. 
\section{}
Граф $G$ имеет множество вершин $V={1,2,3,5,6,10,15,30}$. Граф $G$ содерожит ребро $\{u,v\}$ $(u<v)$, если $v$ делится на $u$ и не существует вершины $s\in V$, такой что и $v$ делится на $s$ и $s$ делится на $u$.\\\\
1. Постройте граф $G$\\\\
\xymatrix{
&1 \ar@{->}[dd]\ar@{->}[dl]\ar@{->}[rr] && 5\ar@{->}[dd]\ar@{->}[dl]\\
3 \ar@{->}[dd]\ar@{->}[rr] & & 15\ar@{->}[dd]\\
& 2 \ar@{->}[rr]\ar@{->}[dl] & &10\ar@{->}[dl]\\
6 \ar@{->}[rr] & & 30}\\\\\\
2. Изоморфен ли этот граф булеву кубу $B_3$, если да, то укажите биекцию\\\\
Булев куб $B_3$:\\\\
\xymatrix{
&000 \ar@{-}[dd]\ar@{-}[dl]\ar@{-}[rr] && 001\ar@{-}[dd]\ar@{-}[dl]\\
010 \ar@{-}[dd]\ar@{-}[rr] & & 011\ar@{-}[dd]\\
& 100 \ar@{-}[rr]\ar@{-}[dl] & &101\ar@{-}[dl]\\
110 \ar@{-}[rr] & & 111\ar@{-}[d]}\\\\\\
Да изоморфен, пусть вершина булева куба $\overline{abc}$, тогда $\overline{abc}\rightarrow2^{a}\cdot3^{b}\cdot5^{c}$
\end{document}