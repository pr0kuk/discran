\documentclass{article}
\usepackage{amsfonts}
\usepackage{amssymb}
\usepackage[all]{xy}
\usepackage[utf8]{inputenc}
\usepackage{amsmath}
\usepackage{qtree}
\usepackage[T2A]{fontenc}
\usepackage[russian]{babel}
\usepackage{mathtext}
\title{Task №8}
\author{Щербаков Алексей Б01-908}
\date{5 November 2019}
\begin{document}
\maketitle
\section{}
Робот ходит по координатной плоскости. На каждом шаге он может увеличить одну координату на 1 или обе координаты на 2. Сколько есть способов переместить Робота из точки (0;0) в точку (4;5)?\\\\
Пусть количество способов попасть из (0;0) в (x;y) равно P(x;y)\\
Тогда $P(x;y) = P(x-2;y-2) + P(x-1;y) + P(x;y-1)\\$
Заполним таблицу, где число на пересечении столбцов с номером x и y - P(x;y)\\\\
\xymatrix{
x/y&0&1&2&3&4&5\\
0  &1&1&1&1&1&1\\
1  &1&2&3&4&5&6\\
2  &1&3&7&12&18&25\\
3  &1&4&12&26&47&76\\
4  &1&5&18&47&101&189\\
}\\\\
\\В ячейке (4;5) число 189, значит P(4;5) = 189.\\
Ответ: 189
\section{}
В магазине продаётся 10 видов пирожных. Сколькими способами можно купить 100 пирожных (порядок покупки не важен)?\\\\
Задача аналогична задаче 4., разобранной на семинаре.  $x_1+x_2+x_3+...+x_{10}=100$ - найти число решений в неотрицательных целых числах. На семинаре была получена формула для количества решений такого уравнения: C$^{k-1}_{n+k-1}=\text{C}^{9}_{109}$\\
\\Алгоритм получения формулы: \\Есть $n+k-1$ мест под $n$ единиц и $k-1$ перегородок. \\Нужно выбрать из $n+k-1$ мест $k-1$ под перегородки: C$^{k-1}_{n+k-1}$\\
\\Ответ: C$^{9}_{109}$
\section{}
Какое слагаемое в разложении $(1+2)^n$ по формуле бинома Ньютона будет наибольшим?\\\\
Слагаемое $A_j = \text{C}^j_n \cdot 2^j$\\
Рассмотрим $\frac{A_j}{A_{j+1}}=\frac{\text{C}^j_n \cdot 2^j}{\text{C}^{j+1}_n \cdot 2^{j+1}}=\frac{\text{C}^j_n}{\text{C}^{j+1}_n \cdot 2}=\frac{j+1}{2n-2j}\ast$\\
$\frac{j+1}{2n-2j} < 1 \rightarrow j+1 < 2n - 2j \rightarrow 3j < 2n - 1 \rightarrow j < \frac{2n-1}{3}$\\
Пока $j < \frac{2n-1}{3}$ слагаемые будут увеличиваться, а после уменьшаться.
Так как $2n-1$ может быть нецелым, то возьмём ближайшее целое число, большее $2n-1$: $k = [\frac{2n-1}{3}]+1$, тогда $A_k = \text{C}^k_n \cdot 2^k = \text{C}^{[\frac{2n-1}{3}]+1}_n \cdot 2^{[\frac{2n-1}{3}]+1}$
\\\\
$\ast$ Формула не работает для $n=j: [\frac{2n-1}{3}]+1 = n\\$
$n-1\leq\frac{2n-1}{2}<n \rightarrow 3n-3\leq2n-1<3n \rightarrow n \in (-1;2]$\\
Т.к. $n$ целое и неотрицательное, то нужно отдельно проверить случаи\\
$n=0$: максимальное первое слагаемое (1)\\
$n=1$: максимальное второе слагаемое (2)\\
$n=2$: максимальное второе слагаемое (4)
\\\\Ответ: \\
Для $n\geq2$: Слагаемое под номером $[\frac{2n-1}{3}]+1$\\
Для $n=1$: Слагаемое под номером $2$\\
Для $n=0$: Слагаемое под номером $1$
\section{}
Найти число слов длины $n$ над алфавитом \{0,1\}, в которых нет двух единиц подряд.\\
\\
Найдём количество способов построить слово длины $k$.
Если последний символ $0$, то количество способов построить такое слово такое же как и слово $k-1$\\
Если последний символ $1$, то предпоследний $0$, тогда, аналогично, способов столько же сколько и у $k-2$\\
Так как на конце может стоять только $0$ или только $1$, то \\$P(k)=P(k-1)+P(k-2)$
\\
\begin{equation*}
 \begin{cases}
   P(0)=1\\
   P(1)=2\\
   P(k)=P(k-1)+P(k-2)\\
 \end{cases}
\end{equation*}
Из системы видно, что $P(n) = F_{n+2}$, где $F_{n+2}$ - число Фибоначчи с номером $n+2$.\\\\
Ответ: $P(n) = F_{n+2}$, где $F_{n+2}$ - число Фибоначчи с номером $n+2$.
\section{}
Комбинаторно доказать:\\\\
1) $\text{C}^m_n\cdot\text{C}^k_m = \text{C}^k_n\cdot\text{C}^{m-k}_{n-k}$\\\\ $\text{C}^m_n\cdot\text{C}^k_m$ - Количество способов из $n$ элементов выбрать $m$, а из них ещё выбрать $k$. Это то же самое, что $\text{C}^{n-m}_n\cdot\text{C}^{k}_{m}$ - Количество способов выбрать $n-m$ элементов из $n$, а из осташивхся выбрать $k$.\\\\
$\text{C}^k_n\cdot\text{C}^{m-k}_{n-k}$ - Количество способов из $n$ выбрать $k$ элементов, а из оставшихся выбрать $m-k$. Это то же самое, что $\text{C}^k_n\cdot\text{C}^{n-m}_{n-k}$ - Количество способов выбрать из $n$ элементов $k$, а из оставшихся $n-m$\\\\
В первом случае мы сначала выбрали $n-m$, а потом $k$ из оставшихся. Во втором случае мы сначала выбрали $k$, а из оставшихся $n-m$, что одно и то же. чтд\\\\\\\\
2) C$^m_n=\text{C}^{m}_{n-2}+2\text{C}^{m-1}_{n-2}+\text{C}^{m-2}_{n-2}$
\\\\Нужно выбрать $m$ элементов из $n$.
\\
В правой части элементы всегда выбираются из $n-2$.\\
Пусть множество всех элементов - $N$, а множество выбранных $M$.
Рассмотрим произвольные элементы $A,B\in N$.\\
Существует 3 варинта:\\
$A,B\in M$: $\text{C}^{m-2}_{n-2}$\\
$A \in M, B \notin M$: $\text{C}^{m-1}_{n-2}+\text{C}^{m-1}_{n-2}$\\
$A, B \notin M$: $\text{C}^{m}_{n-2}$\\
Итого: $\text{C}^{m}_{n-2}+2\text{C}^{m-1}_{n-2}+\text{C}^{m-2}_{n-2}$ чтд.
\section{}
Какое из чисел больше $\text{C}^{F_{998}+1}_{F_{1000}}$ или $\text{C}^{F_{999}+1}_{F_{1000}}$\\\\
$\frac{F_{1000}!}{(F_{998}+1)!(F_{1000}-F_{998}-1)!}$ или $\frac{F_{1000}!}{(F_{999}+1)!(F_{1000}-F_{999}-1)!}$\\
\\$(F_{998}+1)!(F_{1000}-F_{998}-1)!$ или $(F_{999}+1)!(F_{1000}-F_{999}-1)!$\\\\
$(F_{998}+1)!(F_{999}-1)!$ или $(F_{999}+1)!(F_{998}-1)!$\\\\
$(F_{998})!(F_{998}+1)\frac{(F_{999})!}{F_{999}}$ или $(F_{999})!(F_{999}+1)\frac{(F_{998})!}{F_{998}}$\\\\
$(F_{998}+1)F_{998}<(F_{999}+1)F_{999}$\\\\
Значит $\text{C}^{F_{998}+1}_{F_{1000}} > \text{C}^{F_{999}+1}_{F_{1000}}$\\\\\\
Ответ: $\text{C}^{F_{998}+1}_{F_{1000}} > \text{C}^{F_{999}+1}_{F_{1000}}$
\section{}
Комбинаторно доказать $\sum\limits_{k=0}^{(n+1)/2}$ C$^{k}_{n-k+1} = F_{n+2}$\\\\
C$^{k}_{n-k+1}$ - количество способов вставить $k$ единиц между $n-k$ нулями (количество слов, в которых ровно k единиц и никакие из них не стоят рядом).\\
Таким образом, слева - количество слов длины $n$ над алфавитом \{0,1\}, где никакие две единицы не стоят рядом.\\\\
Справа по номеру 4) $F_{n+2}$ - число слов длины $n$ над алфавитом \{0,1\}, в которых нет двух единиц подряд.\\
$k$ - количество единиц в слове. $n-k$ - количество нулей соответственно.
\\\\
Справа и слева выражения тождественно равны чтд.
\section{}
Сколько способов разместить 20 различных книг на 5 полках, если каждая полка может вместить все 20 книг? Размещения, отличающиеся порядком книг на полках, считаются различнами.\\\\
У нас есть 20 книг и 4 разделителя\\
Если бы порядок полок имел значение, то количество способов было бы 24!\\
Так как порядок полок значение не имеет, то нужно разделить на количеству перестановок 4х разделителей (4!)\\\\
Ответ: $\frac{24!}{4!}$
\section{}
Студсовет из 8 человек выбирает из своего состава председателя путём тайного голосования. Каждый может отдать один голос за любого члена студсовета. Результат голосования - число голосов, отданных за каждого кандидата. Сколько существует различных результатов голосования?\\\\
Пусть $x_j$ - количество голосов за кандидата с номером $j$\\
Тогда нужно найти количество целых неотрицательных решений $x_1+x_2+...+x_8=8$\\
Аналогично номеру 2 их $\text{C}^7_{15}$
\\\\Ответ: C$^7_{15}$
\section{}
Сколькими способами можно переставить буквы в слове "ОБОРОНОСПОСОБНОСТЬ", так чтобы две буквы "О" не стояли рядом\\\\
Найдём количество способов переставить буквы без О. $\frac{11!}{2!\cdot2!\cdot 3!}$
\\Теперь будем вставлять бувы О между этими буквами. Выберем 7 мест между 11 буквами, куда вставить О. Будем выбирать только по одному месту между двумя буквами, поэтому две О не будут стоять рядом. Между 11 буквами есть 12 мест. Тогда количество способов выбрать 7 мест из 12: C$^{7}_{12}$. Итого: $\frac{11!\cdot\text{C}^{7}_{12}}{2!\cdot2!\cdot 3!}$
\\\\Ответ: $\frac{11!\cdot\text{C}^{7}_{12}}{24}$

\end{document}