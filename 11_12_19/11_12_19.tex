\documentclass{article}
\usepackage{amsfonts}
\usepackage{amssymb}
\usepackage[all]{xy}
\usepackage[utf8]{inputenc}
\usepackage{amsmath}
\usepackage[T2A]{fontenc}
\usepackage[russian]{babel}
\usepackage{mathtext}
\title{Task №9}
\author{Щербаков Алексей Б01-908}
\date{12 November 2019}
\begin{document}
\maketitle
\section{}
Сколькими способами можно закрасить клетки таблицы $3\times 4$ так, чтобы незакрашенные клетки содержали или верхний ряд, или нижний ряд, или две средние вертикали?\\\\
$n = 2^8 (\text{закрашенный верх}) + 2^8 (\text{закрашенный низ}) + 2^6(\text{закрашенная середина}) - 2^4 (\text{закрашенный верх и низ}) - 2^4 (\text{закрашенный верх и середина}) - 2^4 (\text{закрашенный низ и середина}) + 2^2 (\text{закрашенный верх, низ и середина}) = 532$
\\\\
Ответ: 532
\section{}
Для полёта на Марс набирают группу людей, в которой каждый должен владеть хотя бы одной из профессий повара, медика, пилота или астронома. \\Каждой профессией должно владеть ровно 6 человек. \\Каждой парой ровно 4 человека. \\Каждой тройкой ровно 2. \\Всеми профессиями ровно 1. \\Выполнимо ли задание по сбору такой группы?\\\\
Пусть количество человек в группе $n$\\
$0 = n - 6 \cdot 4 + 4 \cdot 6  - 2 \cdot 4 + 1 \cdot 1 \rightarrow n = 7$\\
Из этих 7 человек 6 владеют профессией $a$. Пусть последний владеет профессией $b$, тогда кроме него ещё пять человек должны владеть профессией $b$. \\Получается, что 5 человек владеют профессиями $a$ и $b$, а таких человек по условию 4. \\Противоречие.
\\\\
Ответ: Невыполнимо.
\section{}
Пусть $A$ и $B$ - конечные непустые множества, и $|A| = n$. Известно, что число инъекций из $A$ в $B$ совпадает с числом сюръекций из $A$ в $B$. Чему равно это число\\\\
Пусть $|B| = m$\\
Если существует инъекция, то $m \geq n$, если существует сюръекция, то $m \leq n$, значит $m=n$\\
Количество инъекций: $\frac{m!}{(m-n)!} = n!$\\\\
Ответ: $n!$
\section{}
В классе 20 учеников, каждый из которых дружит ровно с шестью одноклассниками. Найдите число таких различных компаний из трёх учеников, что в них либо все школьники дружат друг с другом, либо каждый не дружит ни с одним из двух оставшихся.\\\\
Найдём количество троек, где не выполняется условие. В каждой такой тройке есть человек, который дружит с одним и не дружит со вторым. Выберем одного из учеников (20), выберем одного из тех, с кем он дружит (6) и одного из тех, с кем он не дружит (13). Итого: $20\cdot6\cdot13=1560$. Но в этих комбинациях каждый учащийся учтён два раза, поэтому количество таких троек $\frac{1560}{2}=780$\\
Тогда искомых троек C$^3_{20}-780=1140-780=360$\\\\
Ответ: 360
\section{}
Найдите количество неубывающих отображений $f:\{1,2,...,n\} \rightarrow \{1,2,...,m\}$\\\\
Расставим перегородки между значениями функции. Значению между $(i-1)$й перегородкой и $i$й перегородкой соответствует $x_i$, если между перегородками ничего нет, то $x_i$ соответствует предыдущему значению. Так как у нас $n$ аргументов и $m$ значений, то нам нужно выбрать $m-1$ мест под перегородки из $m+n-1$ возможных: C$^{m-1}_{m+n-1}$\\\\
Ответ: C$^{m-1}_{m+n-1}$
\section{}
Чего больше, разбиений $n$-элементного множества на не более чем $k$ подмножеств или разбиений $(n+k)$-элементного множества на ровно $k$ подмножеств?\\\\

\section{}
Сколькими способами можно рассадить за круглым столом $n$ пар влюблённых так, чтобы ни одна пара влюблённых не сидела рядом?\\\\
Предположим, что места пронумерованы...\\
$ans=\sum\limits_{k=0}^n(-1)^{n}N(k)$, $N(k)$ - количество способов рассадить людей так, что $k$ пар влюблённых сидит вместе.\\
Рассмотрим $k$ое слагаемое:\\ Выберем место для первого человека из первой пары $((2n)!)$, второго сажаем на место справа от первого, так как они могут поменяться, то домножим на два $(2\cdot(2n)!)$. Аналогично для второй пары $(2\cdot(2n-2)!)$ и тд.\\
$(2\cdot(2n)!)(2\cdot(2n-2)!)(2\cdot(2n-4)!)...(2\cdot(2n-2k+2)!)=2^k\frac{(2n)!!}{(2k-2)!!}$\\
Умножим на количество вариантов рассадить остальных людей:\\ $N(k)=2^k\frac{(2n)!!}{(2n-2k)!!}(2n-2k)!$\\
$ans=\sum\limits_{k=0}^n(-1)^{n}2^k\frac{(2n)!!}{(2n-2k)!!}(2n-2k)!$
\\Если места стола не пронумерованы, то ответ ещё нужно будет поделить на $2n$\\
Ответ: $\sum\limits_{k=0}^n(-1)^{n}2^k\frac{(2n)!!}{(2n-2k)!!}(2n-2k)!$
\section{}
Конфет - $n$, коробок - $m$\\
Найти число способов разместить конфеты по коробкам:\\
а) конфеты и коробки разные\\
$m^n$\\
б) конфеты одинаковые, коробки разные и непустые\\
Выберем из $n-1$ места под перегородки ровно $m-1$, таким образом получим однозначное разбиение конфет по коробкам: C$^{m-1}_{n-1}$\\
в) конфеты одинаковые, коробки разные\\
Имеем $n+m-1$ место под конфеты и перегородки, выберем из них $m-1$ место под перегородки, остальное заполним конфетами: С$^{m-1}_{n+m-1}$\\
г) конфеты и коробки разные, коробки непустые\\
Представим конфеты в виде аргументов, а коробки в виде значений. Тогда нужно найти количество сюръекций, так как коробки не пустые, а каждой конфете соответствует только одна коробка.\\
$\sum\limits_{k=0}^{m-1}(-1)^k\text{C}^k_m(m-k)^n$\\
д) конфеты разные, коробки одинаковые, коробки непустые\\
Берём ответ из г) и делим на количество перестановок коробок:
$\frac{1}{m!}\sum\limits_{k=0}^{m-1}(-1)^k\text{C}^k_m(m-k)^n$\\
е) конфеты разные, коробки одинаковые\\
Берём ответ из а) и делим на количество перестановок коробок: $\frac{m^n}{m!}$
\end{document}