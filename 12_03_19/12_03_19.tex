\documentclass{article}
\usepackage{amsfonts}
\usepackage{amssymb}
\usepackage[all]{xy}
\usepackage[utf8]{inputenc}
\usepackage{amsmath}
\usepackage[T2A]{fontenc}
\usepackage[russian]{babel}
\usepackage{mathtext}
\title{Task №12}
\author{Щербаков Алексей Б01-908}
\date{03 Dec 2019}
\begin{document}
\maketitle
\section{}
Разложите в ДНФ и КНФ булеву функцию, заданную вектором значений: $f(x,y,z)=00111100$\\\\
\begin{tabular}{ |c|c|c|c| } 
 \hline
 x & y & z & res \\
 0 & 0 & 0 & 0 \\ 
 0 & 0 & 1 & 0 \\ 
 0 & 1 & 0 & 1 \\ 
 0 & 1 & 1 & 1 \\
 1 & 0 & 0 & 1 \\ 
 1 & 0 & 1 & 1 \\
 1 & 1 & 0 & 0 \\ 
 1 & 1 & 1 & 0 \\ 
 \hline
\end{tabular}\\\\
Если $res = 1$, то переменные, равные 1, записываем в ДНФ без изменения, а равные 0 с отрицанием.\\
\\ДНФ: $(\bar x\wedge y\wedge \bar z)\vee (\bar x\wedge y\wedge z) \vee (x\wedge \bar y\wedge \bar z) \vee (x\wedge\bar  y\wedge z)$\\
Если $res = 0$, то переменные, равные 0, записываем в КНФ без изменения, а равные 1 с отрицанием.\\
КНФ: $(x\vee y\vee z)\wedge(x\vee y\vee \bar z)\wedge(\bar x\vee \bar y \vee z) \wedge (\bar x \vee \bar y \vee \bar z)$
$x$ и $y$ существенные, $z$ - фиктивная.
Поэтому $z$ можно не учитывать:
\\ДНФ: $(\bar x\wedge y) \vee (x\wedge \bar y)$\\
КНФ: $(x\vee y)\wedge(\bar x\vee \bar y)$
\\

\section{}
Вычисляется ли константа 0 в базисе $\{\lnot (x_1\rightarrow x_2)\}$
\\\\
\begin{tabular}{ |c|c|c| } 
 \hline
 x_1 & x_2 & res \\
 0 & 0 & 0 \\ 
 0 & 1 & 0 \\ 
 1 & 0 & 1 \\ 
 1 & 1 & 0 \\ 
 \hline
\end{tabular}
\\\\
$\lnot (\lnot (x_1\rightarrow x_2)\rightarrow \lnot (x_1\rightarrow x_2))$\\\\
\begin{tabular}{ |c|c|c| } 
 \hline
 x_1 & x_2 & res \\
 0 & 0 & 0 \\ 
 0 & 1 & 0 \\ 
 1 & 0 & 0 \\ 
 1 & 1 & 0 \\ 
 \hline
\end{tabular}
\\\\Ответ: Да
\section{}
Вычислите MAJ$(x,y,z)$ схемой в базисе Жегалкина $\{1,\wedge,x_1\oplus x_2\}$
\\\\
$MAJ(x,y,z) = (x\wedge y) \oplus (y\wedge z) \oplus (x\wedge z)$\\\\
\xymatrix{
x \ar@{->}[r]  \ar@{->}[ddr]  &  \wedge       \ar@{->}[dr]\\
y \ar@{->}[ur] \ar@{->}[r] & \wedge           \ar@{->}[r]            & \oplus \ar@{->}[r] & \oplus  \\
z \ar@{->}[ur] \ar@{->}[r] & \wedge        \ar@{->}[rru] }\\\\
\section{}
Сколько ненулевых коэффициентов в многочлене Жегалкина, который равен $x_1 \vee x_2 \vee ... \vee x_n$?\\\\
Выразим многочлен в базисе Жегалкин:\\
$x_1 \vee x_2 \vee ... \vee x_n = \lnot(\lnot(x_1 \vee x_2 \vee ... \vee x_n))=\lnot(\bar x_1 \wedge \bar x_2 \wedge ... \wedge \bar x_n)= \lnot((1+x_1)\cdot(1+x_2)\cdot...\cdot(1+x_n))=\lnot(1+A) = 1 + 1 + A = A$\\
$1+A$ содержит $2^n$ одночленов, значит $A$ содержит $2^n-1$ одночлен\\
\\
Ответ: $2^n-1$
\section{}
Докажите полноту базиса \{$x | y$\}\\\\
$\lnot x = x | x$\\
$x \wedge y = \lnot(x | y) = (x | y) | (x | y)$\\
Таким образом с помощью штриха Шеффера можно выразить полный базис $\{\lnot,\wedge\}$, значит штрих Шеффера тоже полный базис
\section{}
Является ли полным базис \{\wedge; \rightarrow\}?\\\\
$1 \wedge\text{ } 1 = 1\\
1 \rightarrow 1 = 1$\\
Значит набор функций замкнут в классе $T_1$, значит не является базисом\\\\
Ответ: Нет
\section{}
Является ли полным базис \{$\lnot; MAJ(x_1,x_2,x_3)$\}?\\\\
$MAJ(x_1,x_2,x_3) = 0, MAJ(\bar x_1,\bar x_2, \bar x_3) = 1$\\
$MAJ(x_1,x_2,x_3)= 1, MAJ(\bar x_1,\bar x_2, \bar x_3) = 0$\\
Функции отрицания и MAJ самодвойственные, следовательно, базис замкнут на $S$.\\\\
Ответ: Нет
\section{}
Пусть $f(x_1,...,x_n)$ - немонотонная функция. Докажите, что $\lnot x_1$ вычисляется в базисе $\{0,1,f\}\\\\$
$\exists A, B: (f(A) = 1) \wedge (f(B) = 0)$ $ \wedge$ $(B$ получается из $A$ заменой любого количества нулей на 1)\\
В наборе $B$ заменим все появившиеся единицы на $x_i$, назовём получившийся набор $C$, тогда если $x_i=1$, то $C=B$ и $f(C)=0$, а если $x_i=0$, то $C=A$ и $f(C)=1$. Таким образом, $f(C)=\lnot x_i$ чтд.
\section{}
Докажите, что всякую монотонную булеву функцию можно вычислить монотонной схемой\\\\
Нужно доказать, что любая монотонная функция представима в ДНФ без отрицаний.
Выпишем таблицу истинности произвольной монотонной функции\\\\
\begin{tabular}{ |c|c|c|c| } 
 \hline
 x & y & z & res \\
 0 & 0 & 0 & 0 \\ 
 0 & 0 & 1 & 0 \\ 
 0 & 1 & 0 & 0 \\ 
 0 & 1 & 1 & 1 \\
 1 & 0 & 0 & 0 \\ 
 1 & 0 & 1 & 1 \\
 1 & 1 & 0 & 1 \\ 
 1 & 1 & 1 & 1 \\ 
 \hline
 \end{tabular}\\\\
 Рассмотрим столбцы, где значение равно 1. Рассмотрим строку где значение равно 1, при этом количество аргументов равное 1 минимально. Т.к. функция монотонная, то существуют строки, равные 1, которые отличаются от этой только заменой одного 0 на 1. Получается $...\vee(\bar a \wedge b \wedge...) \vee (a \wedge b...)..$. Эта дизъюнкция равна 
 ($b \wedge...$). Далее аналогично для любой строки. Таким образом, любую дизъюнкцию конъюнкций с отрицанием можно сократить до конъюнкции без отрицания, значит, любое ДНФ представление любой монотонной функции можно сократить до представления ДНФ без отрицания. чтд.
\section{}
$PAR(x_1,x_2,...,x_n) = 1$, если количество единиц чётное и 0, если нечётно\\\\
а) Выразите $PAR$ через известные булевы функции\\\\
$PAR(x_1,x_2,...,x_n) = x_1 \oplus x_2 \oplus ... \oplus x_n$\\\\
б) При каких $n\geq 1$ можно представить $PAR$ в виде ДНФ без отрицаний\\\\
Функция задаётся ДНФ без отрицаний, только если она монотонная (так как ДНФ без отрицаний монотонна по определению, $(a\wedge b\wedge c\wedge...) \vee (d\wedge e\wedge g\wedge...) \vee... = 1 \rightarrow$ если любой 0 заменить на 1, то значение функции не изменится). Т.к. $PAR$ при $n>1$ немонотонная (если $PAR(A)=1$ и если заменить любой 0 на 1, то значение функции поменяется), значит $n=1$\\
Ответ: 1
\section{}
Докажите, если $f(x_1,...,x_n)$ - нелинейная функция, то $x_1\wedge x_2$ вычисляется в базисе $\{0,1,\lnot,f\}$\\\\
Представим функцию $f$ в форме полинома Жегалкина:\\
Так как функция нелинейная, то существует одночлен, содержащий хотя бы две переменные\\
Представим в форме Жегалкина $f(x_1,x_2,1,1,1,1...) = (x_1\wedge x_2) \oplus x_1 \oplus x_2 \oplus 1$. При этом $(x_1\wedge x_2)$ будет обязательно присутствовать, а оостальные слагаемые не обязательно. Если присутствует только конъюнкция, то $\wedge = f(x_1,x_2,1,1,1,1...)$, если присутствуют ещё члены, то применив отрицание от них можно избавиться.\\
\section{}
Докажите теорему Поста\\\\
Если функции не принадлежит ни одному из классов $T_0, T_1, M, L, S$, то они образуют полный базис.\\
Рассмотрим функию $f(x_1,x_2,...)$\\
Если функция $f \notin T_0$, то $f(x,x,x,x...)=\lnot$ или $1$\\
Если функция $f \notin T_1$, то $f(x,x,x,x...)=\lnot$ или $0$\\
Таким образом из этой функции можно получить отрицание и две константы.\\
По теореме, доказанной в предыдущем задании, мы можем выразить $\wedge$\\
Таким образом, мы из $f$ получили полный базис $\{\lnot, \wedge\}$, значит $f$ так же является полным базисом.
\end{document}